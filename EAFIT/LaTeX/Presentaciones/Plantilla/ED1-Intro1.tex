\documentclass[a4,12pt]{beamer}
%\usetheme{Singapore}
%\usecolortheme{albatross}
\setbeamercovered{transparent=50}

\usepackage{graphicx}
\usepackage{graphics}
\usepackage{rotating}
\usepackage{verbatim} 
\usepackage{amsmath}
\usepackage{floatflt}
\usepackage{subfig}
\usepackage{color}
\usepackage{xcolor}
\usepackage{theorem}
\usepackage{fancyvrb}
\usepackage[utf8]{inputenc}
\usepackage{caption}
%\usepackage{subcaption}

%\usepackage{media9}

%for c syntax
\setbeamercovered{transparent}
\usepackage{listings}
\transdissolve[duration=0.2]
%\usepackage{multimedia}

\captionsetup[subfloat]{labelformat=empty}
\usetheme{texsx}
\definecolor{gray}{RGB}{123, 123, 123}
\definecolor{mrblue}{RGB}{2, 56, 92}

\newcommand{\blue}[1] {\textcolor{mrblue}{#1}}
\newcommand{\red}[1] {\textcolor{gray}{#1}}

\title{Compiladores: \\ Lenguajes Regulares}

\author{Juan F. Cardona}

\date{Departmento de Informática y Sistemas\\ Universidad EAFIT}

\lstset{language=C++, escapeinside={{*@}{@*}},
                basicstyle=\ttfamily,
                keywordstyle=\color{blue}\ttfamily,
                stringstyle=\color{mrbrown}\ttfamily,
                commentstyle=\color{mrblue}\ttfamily,
                morecomment=[l][\color{mrbrown}]{\#}
}



\begin{document}
\frame{\titlepage}



\begin{frame}
\frametitle{Filosofía de vida}

\blue{``Las nuevas tecnologías acabaron con el monopolio del saber del maestro. Hoy la información
está toda allá arriba. A un click, tú puedes bajar toda la información que quieras, en tiempo real. Yo
creo que nunca antes la humanidad había vivido un momento tan apasionante. ¿Entonces cuál
es el papel del maestro ahora? Es un compañero de viaje al conocimiento.''} -- Juan Luis Mejía, Eafit.

\end{frame}


\begin{frame}
\frametitle{Tipos de problemas}
\begin{itemize}
\item \blue{Sorting}

\item \blue{Searching}

\item \blue{Graph problems}

\item \red{String processing}

\item Numerical problems

\item Geometric problems

\item \red{Combinatorial problems}

\end{itemize}

\end{frame}







\end{document}